\documentclass[a4paper, 12pt]{article}

% Подключение русского языка.
\usepackage[russian]{babel}
\usepackage[utf8]{inputenc}
\usepackage[T2A]{fontenc}

% Настройка внешнего вида заголовков.
\usepackage{titlesec}
\newcommand{\chapterFont}{\fontsize{16}{15} \selectfont}
\newcommand{\chapterNumberFont}{\fontsize{14}{15} \selectfont}
\newcommand{\sectionFont}{\fontsize{14}{15} \selectfont}
\renewcommand\thesection{\thechapter.\arabic{section}.}
\titleformat{\chapter}[display]
{\normalfont \chapterFont \bfseries}{\centering \chapterNumberFont
    \chaptertitlename \ \thechapter}{10pt}{\centering \chapterFont}
\titlespacing{\chapter}{0pt}{8pt}{25pt}
\titleformat{\section}
{\normalfont \sectionFont \bfseries}{\centering \thesection}{6pt}
{\centering \sectionFont}
\titlespacing{\section}{0pt}{25pt}{5pt}
% Настройка размеров страниц.
\setlength{\parindent}{0pt} \renewcommand{\baselinestretch}{1.2} \topmargin = 0mm \textwidth =
175mm \textheight = 260mm \hoffset = -17.5mm \voffset = -25.5mm
% Подключение библиотек для изображений.
\usepackage{tcolorbox}
\usepackage{graphicx}
\usepackage{float}
% Подключение библиотек для математических символов.
\usepackage{amsthm}
\usepackage{parcolumns}
\usepackage{amsmath}
\usepackage{amsfonts}
% Настройка внешнего вида для теорем, утверждений и т.д.
\theoremstyle{definition}
\newtheorem*{theorem}{Теорема}
\newtheorem*{definition}{Определение}
\newtheorem*{example}{Пример}
\newtheorem*{remark}{Замечание}
\newtheorem*{solution}{Решение}

\begin{document}

\begin{example}

    $$ f(x_1, x_2, x_3) = 3x_1^2 + 3x_2^2 + 3x_3^2 + 6x_1x_2 - 6x_1x_3 - x_2x_3 = $$
    $$ = (3x_1^2 + 66x_1x_2 - 6x_1x_3) + 3x_2^2 + 3x_3^2 - x_2x_3 =  $$
    $$ = [3(x_1^2 + 2x_1x_2 - 2x_1x_3)] + 3x_2^2 + 3x_3^2 - x_2x_3 = $$
    $$ = [3(x_1 + 2x_1(x_2 - x_3) + (x_2 - x_3)^2)] - 3(x_2 - x_3)^2 + 3x_2^2 + 3x_3^2 - x_2x_3 = $$
    $$ [3(x_1 + x_2 - x_3)^2] + 6x_2x_3 - x_2x_3 = 3(x_1 + x_2 - x_3)^2 + 5x_2x_3 $$

    Замена: $ \begin{cases}
            y_1 = x_1 + x_2 - x_3 \\
            y_2 = x_2             \\
            y_3 = x_3             \\
        \end{cases} $

    $$ 3y_1^2 + 3y_2y_3  $$

    Замена: $ \begin{cases}
            y_1 = x_2       \\
            y_2 = z_2 - z_3 \\
            y_3 = z_2 + z_3 \\
        \end{cases} $

    $$ 3z_1^2 + 5z_2^2 - 5z_3^2 $$
    Пусть $x = S_1 y$ и $y = S_2 z \Rightarrow x = S_1 S_2 z = S z$

    $ \begin{cases}
            x_1 = y_1 - y_2 + y_3 \\
            x_2 = y_2             \\
            x_3 = y_3
        \end{cases}$ и $ S_1 = \begin{pmatrix}
            1 & -1 & 1 \\
            0 & 1  & 0 \\
            0 & 0  & 1 \\
        \end{pmatrix} \cdot \begin{pmatrix}
            1 & 0 & 0  \\
            0 & 1 & -1 \\
            0 & 1 & 1  \\
        \end{pmatrix} $
\end{example}
\par
\begin{example}
    $$ f(x_1, x_2, x_3) = x_1 x_2 + x_1 x_2 + x_2 x_3 = $$
    $$ = t_1^2 - t_2^2 + t_1t_3 +t_2t_3 + t_1t_3 - t_2t_3 = t_1^2 - t_2^2 + 2t_1 t_3 = (t_1 + t_3)^2 - t_2^2 - t_3^2 = $$
    $$ S_1: \begin{cases}
            x_1 = t_1 + t_2 \\
            x_2 = t_1 - t_2 \\
            x_3 = t_3
        \end{cases} \Rightarrow S_1 = \begin{pmatrix}
            1 & 1  & 0 \\
            1 & -1 & 0 \\
            0 & 0  & 1 \\
        \end{pmatrix} \text{ и } S_2: \begin{cases}
            z_1 = t_1 + t_3 \\
            z_2 = t_2       \\
            z_3 = t_3       \\
        \end{cases} \Rightarrow S_2 = \begin{pmatrix}
            1 & 0 & -1 \\
            0 & 1 & 0  \\
            0 & 0 & 1  \\
        \end{pmatrix}$$

    $$ S_1 \cdot S_2 = \begin{pmatrix}
            1 & 1  & 0 \\
            1 & -1 & 0 \\
            0 & 0  & 1 \\
        \end{pmatrix} \cdot \begin{pmatrix}
            1 & 0 & -1 \\
            0 & 1 & 0  \\
            0 & 0 & 1  \\
        \end{pmatrix} = \begin{pmatrix}
            1 & 1  & -1 \\
            1 & -1 & -1 \\
            0 & 0  & 1
        \end{pmatrix} $$

\end{example}

\section*{Привидение к каноничному виду}

    Рассмотрим ДУ 2 порядка:
    $$ \sum_{i, j = 1}^{n} a_{ij} u_{xi} u_{xj} + \sum_{i = 1}^{n} b_i u_{xi} + cu + f = 0 $$
    $$ u = u(x_1, x_2, \ldots, x_n) $$
    $$ \sum_{i = 1}^{n} \tilde{a_i} u_{xi} u_{xi} + \sum_{i = 1}^{n} \tilde{b} u_{xi} + \tilde{c} u + \tilde{f} = 0 $$
    
    Метод Лагранжа
    \begin{definition}
        Для уравнения (1) форма вида $ \sum_{i, j = 1}^{n} a_{ij} x_i x_j $ называется характеристической квадратичной формой. Эту форму нужно привести к каноническому виду $ F = \sum_{i = 1}^{n} \alpha_i x_i^2 $.
    \end{definition}    

    \begin{example}
        Номер 101 \par
        $$ u_{xx} + 2u_{xy} + 2u_{yy} + 4u_{yz} + 5u_{zz} = 0 $$
        $$ x^2 + 2xy + 2yy + 4yz + 5z^2 = 0 $$
        $$ (x^2 + 2xy + y^2) - y^2 + 2y^2 + 4yz + 5z^2 = 0 $$
        $$ (x + y)^2 + y^2 + 4yz + 5z^2 = 0 \Rightarrow  $$
    \end{example} \pagebreak
    \section*{Лекция 1. Классификация и приведение к каноническому виду диференциальных уравнений 2 порядка.}

\begin{definition}
    Уравнением в частных производных 2 порядка называют
    $$ \sum_{i, j = 1}^{n} a_{ij} \dfrac{\delta ^ 2 u}{\delta x_i \delta x_j} + \sum_{i = 1}^{n} b_i \dfrac{\delta u}{\delta x_i} + cu + f(x) = 0 $$
    $$ u = u(x_1, \ldots, x_n), c = c(x_1, \ldots, x_n), a_{ij} = a_{ij} (x_1, \ldots, x_n), b_i = b_i(x_1, \ldots, x_n) $$
\end{definition}
Если $ u, c, a_{ij}, b_{i} = \text{const} \quad \forall i, j $, то
уравнение имеет пстоянные коэфициенты. \par
Рассмотрим случай $n = 2$:
$$ a_{11} (x, y) \dfrac{\delta ^ 2 u}{\delta x^2} + 2 a_{12}(x, y) \dfrac{\delta ^ 2 u}{\delta x \delta y} + a_{22} (x, y) \dfrac{\delta ^ 2 u}{\delta y^2} + F(f(x, y), \dfrac{\delta u}{\delta x}, \dfrac{\delta u}{\delta y}, u) = 0 $$
\begin{remark}
    Принято писать $\dfrac{\delta u}{\delta x} = u_x, \dfrac{\delta ^ 2 u}{\delta x \delta y} = u_{xy}, \dfrac{\delta ^ 2 u}{\delta x ^ 2} = u_{xx}, \ldots. $
\end{remark}

Сделаем замену переменных: $ \xi = \phi (x, y), \quad \eta = \psi (x, y) $. И так как они линейно независимы:
$$ \mathcal{J} = \begin{vmatrix}
        \dfrac{\delta \phi}{\delta x} & \dfrac{\delta \phi}{\delta y} \\
        \dfrac{\delta \psi}{\delta x} & \dfrac{\delta \psi}{\delta y}
    \end{vmatrix} \neq 0 $$
$$ u_x = \xi_x u_\xi + \eta_x u_\eta $$
$$ u_{xx} = (\xi_x u_\xi + \eta_x u_\eta)_x = u_{\xi \xi} \xi^2_x + 2u_{\xi \eta} \xi_x \eta_x + u_{\eta \eta} \eta^2_x + u_\xi \xi_{xx} + u_\eta \eta_{xx} $$
$$ u_{yy} = (\xi_y u_\xi + \eta_y u_\eta)_y = u_{\xi \xi} \xi^2_y + 2u_{\xi \eta} \xi_y \eta_y + u_{\eta \eta} \eta^2_y + u_\xi \xi_{y} + u_\eta \eta_{yy} $$
$$ u_{xy} = (u_\xi \xi_x + u_\eta \eta_x)_y = \ldots. $$
Подставляя это в исходное уравнение, мы получим:
$$ u_{\xi \xi} \overbrace{[a_{11}(\xi_x)^2 + a_{12} \xi_x \xi_y + a_{22} (\xi_y)^2]}^{\tilde{a}_{11}} + u_{\eta \eta} \overbrace{[a_{11} (\eta_x)^2 + 2 a_{12} \eta_x \eta_y + a_{22} (\eta_y)^2]}^{\tilde{a}_{22}} + $$
$$ + u_{\eta \xi} \overbrace{[2 a_{11} \xi_x \eta_x + 2 a_{22} \xi_y \eta_y + 2a_{12} ( \xi_x \eta_y + \xi_y \eta_x )]}^{\tilde{a}_{12}} + \tilde{F} (\xi, \eta, u, u_\xi, u_\eta) = 0 $$
Итого, получаем:
$$ a_{11} (\xi_x)^2 + 2 a_{12} \xi_x \xi_y + a_{22} (\xi_y)^2 = 0 $$
$$ a_{11} \left(\dfrac{\xi_x}{\xi_y}\right)^2 + 2 a_{12} \left(\dfrac{\xi_x}{\xi_y}\right) + a_{22} = 0 $$
Пусть $ \xi(x, y) = \text{const} $, выразим тогда $ y = y(x) $. Отсюда $ d\xi = \xi_x dx + \xi_y dy \Rightarrow \dfrac{dy}{dx} = -\dfrac{\xi_x}{\xi_y} $
Значит имеем:
$$ a_{11} \left(\dfrac{dy}{dx}\right)^2 + 2a_{12} \left(\dfrac{dy}{dx}\right) + a_{22} = 0 $$
\begin{definition}
    $a_{11} (dy)^2 - 2 a_{12} dy dx + a_{22} (dx)^2 = 0$ - характеристическое уравнение исходного уравнения. 
\end{definition}
$$ \dfrac{dy}{dx} = \dfrac{a_{12} \pm \overbrace{\sqrt{(a_{12})^2 - a_{11} a_{22}}}^{D}}{a_{11}} $$
\begin{enumerate}
    \item D > 0 - гиперболический тип;
    \item D = 0 - параболический тип;
    \item D < 0 - эллиптический тип.
\end{enumerate}
В 1. делаем замену $ \xi = \phi (x, y) = c, \quad \eta = \psi (x, y) = c $ \par
В 2. $\xi = \phi (x, y) = c \quad \quad \eta = \psi (x, y) = c$ - выбираем сами. Она должна быть лин. независимо от $\phi (x, y)$ \par
В 3. имеем $ \overbrace{\phi (x, y)}^{\xi} \pm i \overbrace{\psi (x, y)}^{\eta} = c $

В зависимости от $D$ определяется тип дифференциального уравнения.

\begin{remark}
    Виды канонических дифференциальных уравнений второго порядка:
    \begin{itemize}
        \item $D > 0$ — \textit{гиперболический} тип;
        \item $D = 0$ — \textit{параболический} тип;
        \item $D < 0$ — \textit{эллиптический} тип.
    \end{itemize}
\end{remark}

Для каждого из них определён свой порядок действий:
\begin{enumerate}
    \item $D > 0 \implies$ делаем замену $\xi = \phi(x, y) = \text{const}$, $\eta = \psi(x, y) = \text{const}$;
    \item $D = 0 \implies$ у нас единственное решение $\xi = \varphi(x, y) = \text{const}$. Тогда вводим вторую, линейно независимую с $\varphi(x, y)$ функцию $\eta = \psi(x, y) = \text{const}$;
    \item $D < 0 \implies$ у нас будет пара сопряжённых комплексных чисел $\underbrace{\varphi(x, y)}_{\xi(x, y)} \pm i \underbrace{\psi(x, y)}_{\eta(x, y)} = \text{const}$.
\end{enumerate}

\begin{remark}
    Где $\varphi(x, y)$ и $\psi(x, y)$ — первые интегралы системы.
\end{remark}

\subsection*{Рассмотрим каждый из них по отдельности детальнее}

\subsubsection*{Гиперболический тип ($D > 0$)}
Тогда $\xi = \dfrac{\varphi + \psi}{2}$, $\eta = \dfrac{\varphi - \psi}{2}$, а исходное уравнение имеет вид
$$ 2 \tilde{a}_{12} u_{\xi \eta} + \tilde{F} = 0. $$

\subsubsection*{Параболический тип ($D = 0$)}
Тогда $\xi = \varphi(x, y)$, $\eta = \psi(x, y)$, а $D = a_{12}^2 - a_{11} a_{22} = 0 \implies a_{12} = \sqrt{a_{11} a_{22}}$. Собирая полные квадраты, получаем:
$$
\begin{matrix}
\tilde{a}_{11} = (\sqrt{a_{11}} \xi_x + \sqrt{a_{22}} \xi_y)^2 = (-\sqrt{a_{22}} \xi_y + \sqrt{a_{22}} \xi_y) = 0, \\
\tilde{a}_{12} = \sqrt{a_{11}} \xi_x (\sqrt{a_{11}} \eta_x + \sqrt{a_{22}} \eta_y) + \sqrt{a_{22}} \xi_y (\sqrt{a_{11}} \eta_x + \sqrt{a_{22}} \eta_y) = \\
= (\sqrt{a_{11}} \xi_x + \sqrt{a_{22}} \xi_y)(\sqrt{a_{11}} \eta_x + \sqrt{a_{22}} \eta_y).
\end{matrix}
$$

\subsubsection*{Эллиптический тип ($D < 0$)}
Тогда $\xi = \varphi + i \psi$, $\eta = \varphi - i \psi$, а $D = a_{12}^2 - a_{11} a_{22} < 0$. Для исследования уравнений этого типа введём следующие переменные:
$$
\begin{cases}
\alpha = \dfrac{\xi + \eta}{2}, \\
\beta = \dfrac{\xi - \eta}{2i},
\end{cases}
\implies
\begin{matrix}
\xi = \alpha + i \beta = \xi(\alpha(x, y), \beta(x, y)), \\
\eta = \alpha - i \beta = \eta(\alpha(x, y), \beta(x, y)).
\end{matrix}
$$
Найдём тогда $\tilde{a}_{11}$:
$$
\tilde{a}_{11} = a_{11} (\alpha_x + i \beta_x)^2 + 2 a_{12} (\alpha_x + i \beta_x)(\alpha_y + i \beta_y) + a_{22} (\alpha_y + i \beta_y)^2 = 0 \implies
$$
$$
\implies
\begin{cases}
\alpha: a_{11} (\alpha_x)^2 + 2 a_{12} \alpha_x \alpha_y + a_{22} (\alpha_y)^2, \\
\beta: a_{11} (\beta_x)^2 + 2 a_{12} \beta_x \beta_y + a_{22} (\beta_y)^2, \\
i: 2i [a_{11} \alpha_x \beta_x + a_{12} (\alpha_x \beta_y + \beta_x \alpha_y) + a_{22} \alpha_y \beta_y].
\end{cases}
$$
\textbf{Это достигается тогда, когда $\tilde{a}_{11} = \tilde{a}_{22}$, а $\tilde{a}_{12} = 0$}, если положить в формулах $\tilde{a}$ члены $\xi$ и $\eta$ равными соответствующими $\alpha$ и $\beta$.

\begin{example}
    Приведём к канонической форме уравнение $x^4 u_{xx} - y^4 u_{yy} = 0$. Для начала составим характеристическое уравнение:
    $$ x^4 (dy)^2 - y^4 (dx)^2 = 0. $$
    Тогда $D = 2 y^2 > 0 \implies$ \textit{гиперболический тип}. Получаем:
    $$ \left( \dfrac{dy}{dx} \right)^2 - \left( \dfrac{y}{x} \right)^4 = 0 \implies \dfrac{dy}{dx} = \pm \left( \dfrac{y}{x} \right)^2. $$
    Решив дифференциальное уравнение, мы получаем два корня:
    \begin{itemize}
        \item $\xi = \dfrac{1}{y} + \dfrac{1}{x} = C = \text{const}$,
        \item $\eta = \dfrac{1}{y} - \dfrac{1}{x} = C = \text{const}$.
    \end{itemize}
    Далее находим:
    $$ u_{xx} = \left( u_\xi \left( -\dfrac{1}{x^2} \right) + u_\eta \left( \dfrac{1}{x^2} \right) \right)_x = \dfrac{1}{x^4} (u_{\xi \xi} - 2 u_{\xi \eta} + u_{\eta \eta}) + \dfrac{2}{x^3} (u_\xi - u_\eta), $$
    $$ u_{yy} = \left( u_\xi \left( -\dfrac{1}{y^2} \right) + u_\eta \left( -\dfrac{1}{y^2} \right) \right)_y = \dfrac{1}{y^4} (u_{\xi \xi} + 2 u_{\xi \eta} + u_{\eta \eta}) + \dfrac{2}{y^3} (u_\xi + u_\eta). $$
    Следующим шагом вычисляем $x^4 u_{xx} - y^4 u_{yy}$ и получаем:
    $$ x^4 u_{xx} - y^4 u_{yy} = \dots = -4 u_{\xi \eta} + 2 (x - y) u_\xi - 2 (x + y) u_\eta. $$
    После, исходя из формул $\xi$ и $\eta$, выразим $x + y$ и $x - y$ через $\xi$ и $\eta$:
    \begin{itemize}
        \item $x - y = 4 \dfrac{\eta}{\xi^2 - \eta^2}$,
        \item $x + y = 4 \dfrac{\xi}{\xi^2 - \eta^2}$.
    \end{itemize}
    Осталось лишь подставить $x + y$ и $x - y$ и получить уравнение, записанное в канонической форме, зависящее лишь от $\xi$ и $\eta$:
    $$ -\dfrac{8 \eta}{\xi^2 - \eta^2} u_\xi - 4 u_{\xi \eta} - \dfrac{8 \xi}{\xi^2 - \eta^2} u_\eta = 0. $$
\end{example}

\begin{remark}
    На линейной алгебре было нечто подобное для уравнений с постоянными коэффициентами. Наш же способ работает и в том случае, но не наоборот.
\end{remark}

\subsection*{Случай с постоянными коэффициентами}

\subsubsection*{Гиперболический тип ($D > 0$)}
Решаем уравнение и, после интегрирования обеих частей, получаем:
$$ y_{1,2} = \dfrac{a_{12} \pm \sqrt{a_{12}^2 - a_{11} a_{22}}}{a_{11}} x + C. $$
Тогда наши первые интегралы системы имеют вид:
$$
\begin{cases}
\xi = y - \dfrac{a_{12} + \sqrt{a_{12}^2 - a_{11} a_{22}}}{a_{11}} x, \\
\eta = y - \dfrac{a_{12} - \sqrt{a_{12}^2 - a_{11} a_{22}}}{a_{11}} x,
\end{cases}
$$
где они должны равняться константе $C$. В таком случае каноническая форма будет иметь следующий вид:
$$ 2 \tilde{c} a_{12} u_{\xi \eta} + \tilde{F}(\xi, \eta, u) = 0. $$

\subsubsection*{Параболический тип ($D = 0$)}
Имеем уравнение:
$$ dy = \dfrac{a_{12}}{a_{11}} dx. $$
Отсюда делаем следующую замену:
$$ \xi = y - \dfrac{a_{12}}{a_{11}} x. $$
При этом $\eta$ мы выбираем сами. Например, для удобства возьмём $\eta = y$. В таком случае каноническая форма будет иметь следующий вид:
$$ u_{\xi \xi} + \tilde{c}_1 u_\xi + \tilde{c}_2 u_\eta + \tilde{F}(\xi, \eta, u) = 0. $$

\subsubsection*{Эллиптический тип ($D < 0$)}
Тогда будем иметь уравнение:
$$ y = \dfrac{a_{12} \pm i \sqrt{a_{12}^2 - a_{11} a_{22}}}{a_{11}} x + C. $$
Разделяя на мнимую и вещественную части, мы получаем следующую замену:
$$
\begin{cases}
\xi = y - \dfrac{a_{12}}{a_{11}} x, \\
\eta = \dfrac{\sqrt{a_{12}^2 - a_{11} a_{22}}}{a_{11}} x.
\end{cases}
$$
В таком случае каноническая форма будет иметь следующий вид:
$$ u_{\xi \xi} + u_{\eta \eta} + \tilde{F}(\xi, \eta, u, u_\xi, u_\eta) = 0. $$

\subsubsection*{Примеры}

\begin{example}
    Рассмотрим уравнение $\overbrace{36}^{a_{11}} u_{xx} \overbrace{-12}^{2 a_{12}} u_{xy} + \overbrace{1}^{a_{22}} u_{yy} + 18 u_x - 3 u_y = 0$. После замены и нахождения $u_{xx}, u_{xy}, u_{yy}, u_x, u_y$ получаем \textbf{ответ}:
    $$ u_{\eta \eta} + \dfrac{1}{2} u_\xi = 0. $$
\end{example}

\begin{example}
    Рассмотрим уравнение $\overbrace{5}^{a_{11}} u_{xx} + \overbrace{4}^{2 a_{12} \implies a_{12} = 2} u_{xy} + \overbrace{1}^{a_{22}} u_{yy} + u_x + u_y = 0$. Снова делаем замену, находим $u_{xx}, u_{xy}, u_{yy}, u_x, u_y$ и после подстановки получаем \textbf{ответ}:
    $$ u_{\xi \xi} + u_{\eta \eta} + \dfrac{3}{5} u_\xi + \dfrac{1}{5} u_\eta = 0. $$
\end{example}\pagebreak
\end{document}
