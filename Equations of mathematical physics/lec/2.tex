
\section*{Примеры просейших уравнений математичческой физики}
\subsection*{Уравнения малых поперечных колебаний струны.}
У нас есть струна длины $l$ будет определяться
координатами точке в каждый момент $t$. Перемещние точек сруны мы будем описывть $u(x, t)$ -
величина перемещения точки $x$ в момент времени $t$.
Дальше будем предполагать, что колебания малы, и что:
\begin{enumerate}
    \item Будем пренебрегать квадратом производной $u_x^2 (x, t) << 1$.
    \item Также струна - это гибкая упругая нить. Гибкость означает, что сила натяжения направлена по касательной к профилю стены.
    \item На струну действуют внешние силы, направленные вдоль оси $Ou$.
\end{enumerate}
Теперь найдем длину дуги при малом колебании. Выделим малый участок на отрезке $Ox$: $\Delta x = x_2 - x_1$, тогда его длина равна $S = \int\limits_{x_1}^{x_2} \sqrt{1 + u_x^2(x, t)}dx$ $\Rightarrow$
при поперечных колебаниях удлинение струны отсутствует. По закону Гука делаем вывод, что сила натяжения не меняется со временем.
Но зависит ли сила натяжения от $x$? Посчитаем проекцию силы натяжения $T(x)$ в точке $x_1$ и $x_2$:
$$ \begin{cases}
        T_x(x_2) = T(x_2) \cos \alpha = T(x_2) \dfrac{1}{\sqrt{1 + \tan^2 \alpha}} = T(x_2) \Rightarrow \cos \alpha = 1 \\
        T_u(x_2) = T(x_2) \sin \alpha = T(x_2) \tan \alpha \cos \alpha = T(x_2) u_x(x_2, t)
    \end{cases} $$
Поскольку $T(x_1)$ и $T(x_2)$ противоположно направлены, то $T_x(x_1) = -T_x(x_2)$. Воспользуемся \textbf{правилом Даламбера} (равнодействующая всех сил на тело равна нулю)
$$ T(x_2) - T(x_1) = 0 \Rightarrow T(x, t) = T_0 $$
Выходит, что сила натяжения зависит от \textbf{константы}. \par
Рассматриваем временной промежуток $[t, t + \Delta t]$.
Пусть есть тело с массой $m$. Тогда обозначим через $mv(T)$ количество движения, $F\Delta t$ - импульс силы. Тогда:
$$ m v (t + \Delta t) - m v (t) = F_1 \Delta t + F_2 \Delta t + \ldots + F_n \Delta t $$
Обозначим через $f(x, t)$ плотность распределения силы. $\rho(x)$ - плотность материала струны.
А изменение ддд равна
$$ \int\limits_{x_1}^{x_2} \rho (x) [u_t(x, t + \Delta t) - u_t(x, t)]dx = \int\limits_{t}^{t + \Delta t} T_0 u_x (x_2, t) dt - \int\limits_{t}^{t+ \Delta t} T_0 u_x (x_1, t) + \int\limits_{t}^{t + \Delta t} \int\limits_{x_1}^{x_2} f(x, t)dxdt  $$
Применяем теорему Лагранжа о конечных приращениях мы получили $u_t(x, t) = u_{tt}(x, t') \Delta t, t \in (t, t + \Delta t)$.
$$ \int\limits_{x_{1}}^{x_{2}}\rho(x)u_{tt}(x,t')\Delta tdt=\int\limits_{t}^{t+\Delta t}T_{0}u_{t}(x,t)=u_{\times}(x',t)\Delta xdt+\int\limits_{t}^{t+\Delta t}\int\limits_{x_{1}}^{x_{2}}f(x,t)dx $$
Далее применяем теорему о среднем для интеграла:
$$ p(x'') u_{tt} (x', t'') \Delta t \Delta x = T_0u_{xx}(x', t') + f(x''', t''') \Delta x \Delta t $$
И сокращая $ \Delta x \Delta t $ в пределе при $\Delta x, \Delta t \rightarrow 0$ получаем \textbf{уравнение колебания струны}:
$$ \rho (x) u_{tt}(x, t) = T_0 u_{xx} (x, t) + f(x, t), x \in (0, l), t > 0 $$
Для единственности решения ДУ введем начальное и краевые условия. Начальное условие - Они бывают 2 типо:
\begin{enumerate}
    \item 1-го рода $ \begin{cases}
                  u(0, t) = \mu_1(t) \\
                  u(l, t) = \mu_2(t) \\
              \end{cases} $

\end{enumerate}