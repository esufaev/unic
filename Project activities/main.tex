\documentclass[a4paper, 12pt]{article}

% Подключение русского языка.
\usepackage[russian]{babel}
\usepackage[utf8]{inputenc}
\usepackage[T2A]{fontenc}

% Настройка внешнего вида заголовков.
\usepackage{titlesec}
\newcommand{\chapterFont}{\fontsize{16}{15} \selectfont}
\newcommand{\chapterNumberFont}{\fontsize{14}{15} \selectfont}
\newcommand{\sectionFont}{\fontsize{14}{15} \selectfont}
\renewcommand\thesection{\thechapter.\arabic{section}.}
\titleformat{\chapter}[display]
{\normalfont \chapterFont \bfseries}{\centering \chapterNumberFont
    \chaptertitlename \ \thechapter}{10pt}{\centering \chapterFont}
\titlespacing{\chapter}{0pt}{8pt}{25pt}
\titleformat{\section}
{\normalfont \sectionFont \bfseries}{\centering \thesection}{6pt}
{\centering \sectionFont}
\titlespacing{\section}{0pt}{25pt}{5pt}
% Настройка размеров страниц.
\setlength{\parindent}{0pt} \renewcommand{\baselinestretch}{1.2} \topmargin = 0mm \textwidth =
175mm \textheight = 260mm \hoffset = -17.5mm \voffset = -25.5mm
% Подключение библиотек для изображений.
\usepackage{tcolorbox}
\usepackage{graphicx}
\usepackage{float}
% Подключение библиотек для математических символов.
\usepackage{amsthm}
\usepackage{parcolumns}
\usepackage{amsmath}
\usepackage{amsfonts}
% Настройка внешнего вида для теорем, утверждений и т.д.
\theoremstyle{definition}
\newtheorem*{theorem}{Теорема}
\newtheorem*{definition}{Определение}
\newtheorem*{example}{Пример}
\newtheorem*{remark}{Замечание}
\newtheorem*{solution}{Решение}

\begin{document}

Исходные уравнения движения ракеты:

$$
    x'' = \dfrac{(T - \dfrac{C\rho S v^2}{2})\cos{\theta}}{m} - \dfrac{m'x'}{m}
$$
$$
    y'' = \dfrac{(T - \dfrac{C\rho S v^2}{2})\sin{\theta}}{m} - \dfrac{m'y'}{m} - g
$$

Переход к системе первого порядка:

$$
    \begin{cases}
        v_1 = x  \\
        v_2 = x' \\
        v_3 = y  \\
        v_4 = y' \\
    \end{cases} \Rightarrow  \begin{cases}
        v_1' = v_2                                                                  \\
        v_2' = \dfrac{(T - \dfrac{C\rho S v^2}{2})\cos{\theta}}{m} - \dfrac{m' v_2}{m} \\
        v_3' = v_4                                                                  \\
        v_4' = \dfrac{(T - \dfrac{C\rho S v^2}{2})\sin{\theta}}{m} - \dfrac{m' v_4}{m} - g \\
    \end{cases}, \text{ где: } v = \sqrt{v_2^2 + v_4^2}
$$

\vspace{5cm}


$m = m(t)$ - масса ракеты\par
$v = \sqrt{x'^2 + y'^2}$ - скорость движения\par
$\theta = \arctan{\dfrac{y'}{x'}}$ - угол между касательной к траектории и осью $Ox$\par
$g$ - ускорение силы тяжести\par
$S$ - площадь поперечного сечения ракеты\par
$\rho$ - плотность воздуха\par
$C$ - коэффициент лобового сопротивления ракеты\par
$T = T(t)$ - сила тяги двигателя ракеты


\end{document}
